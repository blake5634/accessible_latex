%%  Test Document for make\_accessible.py
%%

\documentclass[letterpaper]{article}

% Uncomment for bibliog.
%\bibliographystyle{unsrt}
\usepackage{hyperref}
\usepackage{graphicx}

%%%%%%%%%%%%%%%%%%%%%%%%%%%%%%%%%%%%%%%%5
%
%  Set Up Margins
\usepackage[margin=0.5in]{geometry}

%
%        Font selection
%
%\renewcommand{\rmdefault}{ptm}             % Times
%\renewcommand{\rmdefault}{phv}             % Helvetica
%\renewcommand{\rmdefault}{pcr}             % Courier
%\renewcommand{\rmdefault}{pbk}             % Bookman
%\renewcommand{\rmdefault}{pag}             % Avant Garde
%\renewcommand{\rmdefault}{ppl}             % Palatino
%\renewcommand{\rmdefault}{pch}             % Charter


% Make table rows deeper
%\renewcommand\arraystretch{2.0}% Vertical Row size, 1.0 is for standard spacing)

\begin{document}
\title{Test Document for {\tt make\_accessible.py}}

\author{
        Blake Hannaford\\
        Department of Electrical Engineering \\
        The University of Washington
}

\date{\today}

\maketitle

\section{Purpose and Access}
The purpose of this document is for testing a script to make latex documents accessible.
This document is a ``naive'' {\tt latex} file where no tags have been used for accessibility.

The script {\tt make\_accessible.py} was written with the help of Claude Code but
it in itself is just a regular python script and it does not contain or invoke any
AI.

It can be accessed at \href{https://github.com/blake5634/accessible\_latex}{Github}.

\subsection{Suggeted procedure}
\begin{enumerate}
   \item  install the \href{https://pandoc.org/installing.html}{Pandoc package}:
   \item  {\tt run > python3 make\_accessible.py --help} for info.
   \item  try it on your latex files (make backups first!!)
   \item  Go through your new {\tt \_acc.tex} file to write the alternate texts for your {\tt \\includegraphics()} figures.  You can quickly search for {\tt ***!!***} to
   find them.
   \item  Compile your new latex and see how it looks
   \item  Run it through an accessibility checker like the free one at: \href{https://www.accessibilitychecking.com/}{Accessibility Checker.com}
   \item  Thank Claude Code.
\end{enumerate}



%%%%%%%%%%%%%%%%%%%%%%%%%%%%%%%%%%%%%%%%%%%%%%%%%%%%%%%%%%%%%%%%%%%
% Itemize

\begin{itemize}
  \item
  \item
  \item
\end{itemize}

%  Use name of bibliography files without .bib extension
%\bibliography{brl}
\end{document}

